%!TEX root = ../bare_jrnl.tex

\begin{abstract}
	
	

    % We present practical Bayesian inferences and exploration methods for count data collected by autonomous robots with unreliable sensors in human-populated environments. It addresses the problem of drawing incorrect inferences from unreliable count data which affects the effectiveness of robot exploration in maximizing its interaction with humans. Two contributions are presented in this paper: (1) A set of inference methods for a Poisson process which takes into account the unreliability of the robotic sensory systems used to count events is proposed, investigated, and empirically tested. The actual posterior distribution of such Poisson processes is estimated via two tractable approximations. Variations of these processes are presented, in which (i) sensors are uncorrelated, (ii) sensors are correlated, (iii) the unreliability of the observation model, when built from data, is accounted for. (2) Several exploration methods based on optimistic predictions from the resulting posteriors of contribution (1) are proposed, and empirically evaluated. They are assessed by the way they improve the number of human encounters the robot experiences. The results indicate that addressing the unreliable sensors improve human encounters by at least a factor of 2.   

Consider a mobile robot exploring an office building with the aim of observing as much human activity as possible over several days. It must learn where and when people are to be found, count the observed activities, and revisit popular places at the right time. There are two sources and types of uncertainty: human behaviour (aleatoric) and the robot's unreliable sensors (epistemic). By representing both types, the robot can draw better inferences and explore more efficiently. To enable this, we model the activity counts for each time and place as a partially observable Poisson process (POPP). The paper presents extensions to POPP for the following cases: (i) the robot's sensors are correlated, (ii) the robot's sensor model, itself built from data, is also unreliable, (iii) both are combined. Experiments show that the robot makes better inferences and explores more efficiently with the more sophisticated variants.

    % The Poisson assumption is a popular choice when data arises in the form of counts. In many applications, such as in mobile robotics, such counts are prone to systematic error due to noisy sensors and perception algorithms. In this paper we present a collection of Bayesian estimators for the \emph{partially observable} variant of the Poisson process (POPP) which address this problem. The presented estimators deal with cases in which (i) the sensors are uncorrelated, (ii) the sensors are correlated, and (iii) the unreliability of the observation model, when built from data, is accounted for. The resulting posterior is used to drive exploration of a mobile robot with unreliable sensors. We also present an empirical analysis showing the benefit of correcting miscounts in both real and synthetic data, and robot exploration.  
\end{abstract}

\begin{IEEEkeywords}
Poisson processes, partial observability, misclassified counts, robot exploration.
\end{IEEEkeywords}

%Automated control of a stochastic process requires estimates of both aleatoric and epistemic uncertainty. These allow rational choice between exploration to reduce epistemic uncertainty and optimisation of performance in the face of aleatoric uncertainty. This is true for domains as diverse as optimal drug trial design, efficient routing while learning a map, or maximing performance while learning to play a game.